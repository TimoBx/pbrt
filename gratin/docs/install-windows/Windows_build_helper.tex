\documentclass{beamer}

\usepackage[utf8]{inputenc}
\usepackage{default}
\usepackage{hyperref}
\usepackage{color}

\usetheme{Warsaw}
\title[Gratin : Windows Release]{Gratin : Windows Release\\How to build and release Gratin on Windows}
\author{Bourgeas Jean}
\institute{INRIA}
\date{\today}

\setbeamertemplate{navigation symbols}{
\insertframenavigationsymbol % Icône frame
\insertsubsectionnavigationsymbol % Icône sous section
\insertsectionnavigationsymbol % Icône section
\insertdocnavigationsymbol % Icône docnavigation
\insertbackfindforwardnavigationsymbol % Icône backfindforward
}

\AtBeginSection[]
{
  \begin{frame}
    \frametitle{Sommaire}
    \tableofcontents[currentsection]
  \end{frame} 
}

\begin{document}
\begin{frame}
  \titlepage
\end{frame}

\section{Installation}
\subsection{Compiler and CMake}

\begin{frame}
  \begin{enumerate}
    \item If you already have Microsoft visual studio do nothing here.\\
Otherwise, you have to install it or \color{blue}\href{http://www.mingw.org/}{MinGW}\color{black}.
    \item You will now need \color{blue}\href{http://www.cmake.org/}{CMake}\color{black}.\\Get the windows installer in the download section.
\\Once again, you just have to install it. No need to launch it.
  \end{enumerate}
\end{frame}

\subsection{Gratin and Eigen3}
\begin{frame}
  \begin{enumerate}
    \item Get the latest version of \color{blue}\href{http://gratin.gforge.inria.fr/}{Gratin}\color{black}. It should work with any version after v0.3.
    \item You will also need to download \color{blue}\href{http://eigen.tuxfamily.org/index.php?title=Main_Page}{Eigen3}\color{black}.
  \end{enumerate}
\end{frame}

\subsection{Qt and QtCreator}
\begin{frame}
  You will now need to install \color{blue}\href{https://www.qt.io/download/}{QtCreator}\color{black}.\\
During the installation, be sure that QtCreator and the latest version of qt are checked.
\end{frame}

\section{Configuration}
\subsection{CMake}
\begin{frame}
  \begin{enumerate}
    \item To launch QtCreator, go to your Gratin folder. Remove CMakeLists.txt.user if it exists.\\
Then right click on CMakeLists.txt and launch it with QtCreator.
    \item If there is no pack, you will need to create one giving the paths to CMake and your compiler.
    \item When you're at the CMake step, add : -DEIGEN3\_INCLUDE\_DIR="path/to/eigen3" in the argument command line.\\
Just execute CMake and you're done.
  \end{enumerate}
\end{frame}

\subsection{Qt}
\begin{frame}
  \begin{enumerate}
    \item Press Ctrl+5 And be sure that you are compiling in all or release mode, not in debug mode.
    \item Run Gratin. If there is some missing functions, it is possible that your compiler doesn't know them.\\
Add them in src-gratin/misc/extinclude.h at the end of the file. We already did this for log2.\\
It is also possible that dirent.h is missing in your compiler. You can find it online.
  \end{enumerate}
\end{frame}

\subsection{Gratin}
\begin{frame}
  All you need to do now is to add some .dll to your release folder :
  \begin{itemize}
    \item icudt53.dll
    \item icuin53.dll
    \item icuuc53.dll
    \item Qt5Core.dll
    \item Qt5Gui.dll
    \item Qt5Svg.dll
    \item Qt5Widgets.dll
    \item Qt5Xml.dll
  \end{itemize}
  They can be found at : Qt/5.4/msvc2013\_64\_opengl/bin
\end{frame}

\begin{frame}
  You will also need to add some qt plugins folders :
  \begin{itemize}
    \item imageformats
    \item platforms
  \end{itemize}
  They can be found at : Qt/5.4/msvc2013\_64\_opengl/plugins
\end{frame}

\begin{frame}
  You can now execute Gratin by launching gratin.exe. That folder as no dependencies.\\Use it anytime, anywhere. Have fun !
\end{frame}

\end{document}
