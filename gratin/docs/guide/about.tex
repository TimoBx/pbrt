\addsection{Gratin}
%
\addsubsection{Summary}
%
Gratin is a programmable node-based system tailored to the creation,
manipulation and animation of 2D/3D data in real-time on GPUs.
It is written in C++ and uses \href{http://www.qt.io/}{Qt}
for the interface, \href{http://eigen.tuxfamily.org/}{Eigen} for
linear algebra, \href{https://www.opengl.org/}{OpenGL} for renderings
and optionally \href{http://www.openexr.com/}{OpenExr} for loading and
saving high dynamic range images.  It is free and open source
(licensed under MPL v2.0) and relies on OpenGL and GLSL to ensure wide
OS and GPU compatibility.  Source code and installation packages are
available at \url{http://gratin.gforge.inria.fr/}.

\addsubsection{Contribute!}
You implemented a paper in Gratin, or you designed a useful node that
would benefit the graphics community? Please, send it to us, either as
a node (.grac file) or as a pipeline (.gra file). We will be pleased
to integrate it in the next release if possible. The nodes should come
with the author names, the citation of the paper (if there is a paper)
and a small description explaining how to use it.
